%
% This section includes brief examples, including all source
% code and gnuplot images where possible, of a geometric problem,
% a graph problem, and a hypergraph problem.  
%
\chapter{Examples}
\label{cha:ex}

This chapter begins with a very simple example.  We use the Zoltan
library in a parallel application to partition a small set of 
objects, each of which has only a global ID and a weight.

We follow this with three more realistic examples.
These examples use the Zoltan library to apply a geometric method,
a graph method and a then hypergraph method to a small mesh, graph
and hypergraph respectively.

If you have the Zoltan source code, you can find these
examples in the \textbf{examples/C} directory.  The versions
found in the source code include code to
display the initial partitioning followed by Zoltan's computed
partitioning.  It may be helpful to compile and run them, and
then try changing parameters 
to see how the partitioning is affected.  Consult the Zoltan
User's Guide at
\url{http://www.cs.sandia.gov/Zoltan/ug_html/ug.html}
for a complete list of all partitioning methods and all of the
parameters for each partitioning method.

The next four examples are written in C.  At the end of this section,
we also provide a C++ version of the geometric example.

As you read through these examples you will see that, regardless 
of the load balancing method chosen, every program must:

\begin{itemize}
\item initialize the Zoltan library
\item supply load balancing parameters to Zoltan
\item define query functions that Zoltan will use to obtain information from the application about the objects to be partitioned
\item call Zoltan\_LB\_Partition to begin the load balancing calculation
\item free memory allocated by Zoltan when done
\end{itemize}

\clearpage
\section{A very simple partitioning method}

This example is a C program which uses Zoltan to partition
in parallel 36 objects, each of which has a weight.  The goal is
to balance the weights across the parallel application.

It uses the Zoltan method called BLOCK, which simply assignes 
the first $n/p$ data objects to the first process and so on.
This method really exists as an example for developers
to learn how to write partitioning methods.  
It can also be used for testing in applications, but was never
intended to be a serious partitioning strategy for real applications.
However, it is a good choice for our first example.  

If you are a new user of Zoltan, you may want to initially get your
application working with the BLOCK method.  Once you have
that first step accomplished, you can modify it to use one of the
other partitioning methods.

In order to use the BLOCK method, we must define two query functions:

\begin{itemize}
\item A function of type ZOLTAN\_NUM\_OBJ\_FN which provides the number of objects belonging to this process.  In our example we call this function \textbf{get\_number\_of\_objects}.
\item A function of type ZOLTAN\_OBJ\_LIST\_FN which supplies the global IDs and optional weights for the objects belonging to the process.  We name this function \textbf{get\_object\_list}.
\end{itemize}

The \textbf{ZOLTAN\_*} function prototyes named above
are defined in the Zoltan file \textbf{include/zoltan.h}.

Zoltan requires that the parallel application supply a unique 
global ID to identify each of the objects to be partitioned.
Zoltan is very flexible about the data type used to represent
a global ID, and only requires that it be allowed to represent
your global ID internally as an array of integers of some user-defined
length.  Zoltan also permits the use of an additional set of
IDs that are local to a process.  If you supply these optional local IDs,
Zoltan will also include them when querying your application for data.

A version of the 
code listed below may be found in the Zoltan source in file
\textbf{examples/C/simpleBLOCK.c}. 

\begin{flushleft}
\begin{verbatim}
#include <mpi.h>
#include <stdio.h>
#include <stdlib.h>
#include "zoltan.h"

static int numObjects = 36;
static int numProcs, myRank, myNumObj;
static int myGlobalIDs[36];

static float objWeight(int globalID)
{
float w;     /* simulate an initial imbalance */
  if (globalID % numProcs == 0) w = 3;
  else                          w = (globalID % 3 + 1);
  return w;
}
static int get_number_of_objects(void *data, int *ierr)
{
  *ierr = ZOLTAN_OK;
  return myNumObj;
}
static void get_object_list(void *data, int sizeGID, int sizeLID,
            ZOLTAN_ID_PTR globalID, ZOLTAN_ID_PTR localID,
            int wgt_dim, float *obj_wgts, int *ierr)
{
int i;
  *ierr = ZOLTAN_OK;
  for (i=0; i<myNumObj; i++){
    globalID[i] = myGlobalIDs[i];
    if (obj_wgts) obj_wgts[i] = objWeight(myGlobalIDs[i]);
  }
}
int main(int argc, char *argv[])
{
  int rc, i, changes, numGidEntries, numLidEntries, numImport, numExport;
  float ver;
  struct Zoltan_Struct *zz;
  ZOLTAN_ID_PTR importGlobalGids, importLocalGids;
  ZOLTAN_ID_PTR exportGlobalGids, exportLocalGids; 
  int *importProcs, *importToPart, *exportProcs, *exportToPart;

  /******************************************************************
  ** Initialize MPI and Zoltan
  ******************************************************************/

  MPI_Init(&argc, &argv);
  MPI_Comm_rank(MPI_COMM_WORLD, &myRank);
  MPI_Comm_size(MPI_COMM_WORLD, &numProcs);

  rc = Zoltan_Initialize(argc, argv, &ver);

  if (rc != ZOLTAN_OK){
    printf("Error initializing Zoltan\n");
    MPI_Finalize();
    exit(0);
  }

  /******************************************************************
  ** Create a simple initial partitioning for this example
  ******************************************************************/

  for (i=0, myNumObj=0; i<numObjects; i++)
    if (i % numProcs == myRank)
      myGlobalIDs[myNumObj++] = i+1;

  /******************************************************************
  ** Create a Zoltan library structure for this instance of load
  ** balancing.  Set the parameters and query functions.
  ******************************************************************/

  zz = Zoltan_Create(MPI_COMM_WORLD);

  /* General parameters */

  Zoltan_Set_Param(zz, "LB_METHOD", "BLOCK");  /* name of Zoltan method */
  Zoltan_Set_Param(zz, "NUM_GID_ENTRIES", "1"); /* size of global ID    */
  Zoltan_Set_Param(zz, "NUM_LID_ENTRIES", "0"); /* no local IDs         */
  Zoltan_Set_Param(zz, "OBJ_WEIGHT_DIM", "1"); /* weights are 1 float   */

  /* Query functions */

  Zoltan_Set_Num_Obj_Fn(zz, get_number_of_objects, NULL);
  Zoltan_Set_Obj_List_Fn(zz, get_object_list, NULL);

  /******************************************************************
  ** Call Zoltan to partition the objects.
  ******************************************************************/

  rc = Zoltan_LB_Partition(zz, /* input (all remaining fields are output) */
        &changes,        /* 1 if partitioning was changed, 0 otherwise */ 
        &numGidEntries,  /* Number of integers used for a global ID */
        &numLidEntries,  /* Number of integers used for a local ID */
        &numImport,      /* Number of objects to be sent to me */
        &importGlobalGids,  /* Global IDs of objects to be sent to me */
        &importLocalGids,   /* Local IDs of objects to be sent to me */
        &importProcs,    /* Process rank for source of each incoming object */
        &importToPart,   /* New partition for each incoming object */
        &numExport,      /* Number of objects I must send to other processes*/
        &exportGlobalGids,  /* Global IDs of the objects I must send */
        &exportLocalGids,   /* Local IDs of the objects I must send */
        &exportProcs,    /* Process to which I send each of the objects */
        &exportToPart);  /* Partition to which each object will belong */

  if (rc != ZOLTAN_OK){
    printf("Error in Zoltan library\n");
    MPI_Finalize();
    Zoltan_Destroy(&zz);
    exit(0);
  }

  /******************************************************************
  ** Here you would send the objects to their new partitions.
  ******************************************************************/

  /******************************************************************
  ** Free the arrays allocated by Zoltan_LB_Partition, and free
  ** the storage allocated for the Zoltan structure.
  ******************************************************************/

  Zoltan_LB_Free_Part(&importGlobalGids, &importLocalGids, 
                      &importProcs, &importToPart);
  Zoltan_LB_Free_Part(&exportGlobalGids, &exportLocalGids, 
                      &exportProcs, &exportToPart);

  Zoltan_Destroy(&zz);

  MPI_Finalize();

  return 0;
}
\end{verbatim}
\end{flushleft}

\clearpage
\section{Geometric, Graph and Hypergraph Examples}

The simple graph used in all three examples in this section
is illustrated in Figure~\ref{fig:mesh25}.
It is defined in the Zoltan source file \textbf{examples/C/simpleGraph.h},
which is listed in Figure~\ref{fig:simpleGraphDotH}.

\begin{figure}[bottom]
\label{fig:mesh25}
\begin{center}
\begin{verbatim}
   21----22----23----24---25
   |     |     |     |    |
   16----17----18----19---20
   |     |     |     |    |
   11----12----13----14---15
   |     |     |     |    |
   6-----7-----8-----9----10
   |     |     |     |    |
   1-----2-----3-----4----5
\end{verbatim}
\caption{Mesh with 25 vertices}
\end{center}
\end{figure}

\begin{figure}
\label{fig:simpleGraphDotH}
\begin{flushleft}
\begin{verbatim}
#ifndef SIMPLEGRAPH_H
#define SIMPLEGRAPH_H
static int numvertices=25;
static int simpleNumEdges[25] = {
2, 3, 3, 3, 2,
3, 4, 4, 4, 3,
3, 4, 4, 4, 3,
3, 4, 4, 4, 3,
2, 3, 3, 3, 2
};
static int edges[25][4]={
{2,6},       /* adjacent to vertex 1 */
{1,3,7},     /* adjacent to vertex 2 */
{2,8,4},
{3,9,5},
{4,10},
{1,7,11},
{6,2,8,12},
{7,3,9,13},
{8,4,10,14},
{9,5,15},
{6,12,16},
{11,7,13,17},
{12,8,14,18},
{13,9,15,19},
{14,10,20},
{11,17,21},
{16,12,18,22},
{17,13,19,23},
{18,14,20,24},
{19,15,25},
{16,22},
{21,17,23},
{22,18,24},
{23,19,25},
{24,20}      /* adjacent to vertex 25 */
};
#endif
\end{verbatim}
\end{flushleft}
\caption{Adjacencies for a simple mesh}
\end{figure}


\clearpage
\subsection{Recursive Coordinate Bisection}
\label{sec:rcb}

This example uses \emph{Recursive Coordinate Bisection}, 
one of Zoltan's geometric methods, to partition
the vertices of the simple mesh.  This method will use
the coordinates of each vertex, and the weight of each
vertex, in determining a partitioning.  The mesh adjacencies are
irrelevant in this example, although we define the weight of
a vertex to be the number of mesh neighbors it has.

First we will define the query functions required by RCB.
The full source code for this example may be found in
\textbf{examples/C/simpleRCB.c}.

\subsubsection{Application defined query functions}

We must define four query functions.  The function prototyes named 
below are defined in the Zoltan file \textbf{include/zoltan.h}.

\begin{itemize}
\item A function of type ZOLTAN\_NUM\_OBJ\_FN (Figure~\ref{fig:NumObj}) which provides the number of objects belonging to this process 
\item A function of type ZOLTAN\_OBJ\_LIST\_FN (Figure~\ref{fig:ObjList}) which supplies the global IDs and optional weights for the objects belonging to the process
\item A function of type ZOLTAN\_NUM\_GEOM\_FN (Figure~\ref{fig:NumGeom}) which provides the dimension of the objects
\item A function of type ZOLTAN\_GEOM\_MULTI\_FN (Figure~\ref{fig:GeomMulti}) which supplies the coordinates of the objects and optionally their weights
\end{itemize}

An alternative to defining a single ZOLTAN\_OBJ\_LIST\_FN is to define
a ZOLTAN\_FIRST\_OBJ\_FN and a ZOLTAN\_NEXT\_OBJ\_FN.  The first function
would return the global ID for first object.  Each call to the second function would
return a subsequent global ID.

An alternative to ZOLTAN\_GEOM\_MULTI\_FN is to define a ZOLTAN\_GEOM\_FN which
returns the coordinates for a single object rather than for a list
of objects.

The variables referred to in the query functions below were defined in the header
file ~\ref{fig:simpleGraphDotH} to describe the mesh in Figure ~\ref{fig:mesh25}.


\begin{figure}
\begin{flushleft}
\begin{verbatim}
static int get_number_of_objects(void *data, int *ierr)
{
int i, numobj=0;

  for (i=0; i<simpleNumVertices; i++){
    if (i % numProcs == myRank) numobj++;
  }
  *ierr = ZOLTAN_OK;
  return numobj;
}
\end{verbatim}
\end{flushleft}
\caption{A ZOLTAN\_NUM\_OBJ\_FN query function}
\label{fig:NumObj}
\end{figure}

\begin{figure}
\begin{flushleft}
\begin{verbatim}
static void get_object_list(void *data, int sizeGID, int sizeLID,
            ZOLTAN_ID_PTR globalID, ZOLTAN_ID_PTR localID,
                  int wgt_dim, float *obj_wgts, int *ierr)
{
int i, next;

  if ( (sizeGID != 1) || (sizeLID != 1) || (wgt_dim != 1)){ 
    *ierr = ZOLTAN_FATAL;
    return;
  }

  for (i=0, next=0; i<simpleNumVertices; i++){
    if (i % numProcs == myRank){
      globalID[next] = i+1;   /* application wide global ID */
      localID[next] = next;   /* process specific local ID  */
      obj_wgts[next] = (float)simpleNumEdges[i];  /* weight */
      next++;
    }
  }

  *ierr = ZOLTAN_OK;

  return;
}
\end{verbatim}
\end{flushleft}
\caption{A ZOLTAN\_OBJ\_LIST\_FN query function}
\label{fig:ObjList}
\end{figure}

\begin{figure}
\begin{flushleft}
\begin{verbatim}
static int get_num_geometry(void *data, int *ierr)
{
  *ierr = ZOLTAN_OK;
  return 2;
}
\end{verbatim}
\end{flushleft}
\caption{A ZOLTAN\_NUM\_GEOM\_FN query function}
\label{fig:NumGeom}
\end{figure}

\begin{figure}
\begin{flushleft}
\begin{verbatim}
static void get_geometry_list(void *data, int sizeGID, int sizeLID,
                      int num_obj,
             ZOLTAN_ID_PTR globalID, ZOLTAN_ID_PTR localID,
             int num_dim, double *geom_vec, int *ierr)
{
int i;
int row, col;
   
  if ( (sizeGID != 1) || (sizeLID != 1) || (num_dim != 2)){
    *ierr = ZOLTAN_FATAL; 
    return;
  }
    
  for (i=0;  i < num_obj ; i++){
    row = (globalID[i] - 1) / 5;
    col = (globalID[i] - 1) % 5;
  
    geom_vec[2*i] = (double)col;
    geom_vec[2*i + 1] = (double)row;
  }

  *ierr = ZOLTAN_OK;
  return;
} 
\end{verbatim}
\end{flushleft}
\caption{A ZOLTAN\_GEOM\_MULTI\_FN query function}
\label{fig:GeomMulti}
\end{figure}

\clearpage
\subsubsection{Calling the Zoltan library}

The source code below may be found in the file
\textbf{examples/C/simpleRCB.c}.
More information about RCB parameters
may be found in the Zoltan User's Guide at
\url{http://www.cs.sandia.gov/Zoltan/ug_html/ug_alg_rcb.html}.

\begin{flushleft}
\begin{verbatim}
#include <mpi.h>
#include <stdio.h>
#include <stdlib.h>
#include "zoltan.h"
#include "simpleGraph.h"
#include "simpleQueries.h"

int main(int argc, char *argv[])
{
  int rc, i, ngids, nextIdx;
  struct Zoltan_Struct *zz;
  int changes, numGidEntries, numLidEntries, numImport, numExport;
  ZOLTAN_ID_PTR importGlobalGids, importLocalGids;
  ZOLTAN_ID_PTR exportGlobalGids, exportLocalGids; 
  int *importProcs, *importToPart, *exportProcs, *exportToPart;
  float *wgt_list, ver;
  int *gid_flags, *gid_list, *lid_list;

  /******************************************************************
  ** Initialize MPI and Zoltan
  ******************************************************************/

  MPI_Init(&argc, &argv);
  MPI_Comm_rank(MPI_COMM_WORLD, &myRank);
  MPI_Comm_size(MPI_COMM_WORLD, &numProcs);

  rc = Zoltan_Initialize(argc, argv, &ver);

  if (rc != ZOLTAN_OK){
    MPI_Finalize();
    exit(0);
  }
\end{verbatim}
\end{flushleft}

\clearpage
\begin{flushleft}
\begin{verbatim}
  /******************************************************************
  ** Create a Zoltan library structure for this instance of load
  ** balancing.  Set the parameters and query functions that will
  ** govern the library's calculation.  See the Zoltan User's
  ** Guide for the definition of these and many other parameters.
  ******************************************************************/

  zz = Zoltan_Create(MPI_COMM_WORLD);

  /* General parameters */

  Zoltan_Set_Param(zz, "DEBUG_LEVEL", "0");
  Zoltan_Set_Param(zz, "LB_METHOD", "RCB");
  Zoltan_Set_Param(zz, "NUM_GID_ENTRIES", "1"); 
  Zoltan_Set_Param(zz, "NUM_LID_ENTRIES", "1");
  Zoltan_Set_Param(zz, "OBJ_WEIGHT_DIM", "1");
  Zoltan_Set_Param(zz, "RETURN_LISTS", "ALL");

  /* RCB parameters */

  Zoltan_Set_Param(zz, "KEEP_CUTS", "1"); 
  Zoltan_Set_Param(zz, "RCB_OUTPUT_LEVEL", "0");
  Zoltan_Set_Param(zz, "RCB_RECTILINEAR_BLOCKS", "1"); 

  /* Query functions - defined in simpleQueries.h, to return
   * information about objects defined in simpleGraph.h      */

  Zoltan_Set_Num_Obj_Fn(zz, get_number_of_objects, NULL);
  Zoltan_Set_Obj_List_Fn(zz, get_object_list, NULL);
  Zoltan_Set_Num_Geom_Fn(zz, get_num_geometry, NULL);
  Zoltan_Set_Geom_Multi_Fn(zz, get_geometry_list, NULL);

\end{verbatim}
\end{flushleft}

\clearpage
\begin{flushleft}
\begin{verbatim}

  /******************************************************************
  ** Zoltan can now partition the vertices in the simple mesh.
  ** In this simple example, we assume the number of partitions is
  ** equal to the number of processes.  Process rank 0 will own
  ** partition 0, process rank 1 will own partition 1, and so on.
  ******************************************************************/

  rc = Zoltan_LB_Partition(zz, /* input (all remaining fields are output) */
        &changes,        /* 1 if partitioning was changed, 0 otherwise */ 
        &numGidEntries,  /* Number of integers used for a global ID */
        &numLidEntries,  /* Number of integers used for a local ID */
        &numImport,      /* Number of vertices to be sent to me */
        &importGlobalGids,  /* Global IDs of vertices to be sent to me */
        &importLocalGids,   /* Local IDs of vertices to be sent to me */
        &importProcs,    /* Process rank for source of each incoming vertex */
        &importToPart,   /* New partition for each incoming vertex */
        &numExport,      /* Number of vertices I must send to other processes*/
        &exportGlobalGids,  /* Global IDs of the vertices I must send */
        &exportLocalGids,   /* Local IDs of the vertices I must send */
        &exportProcs,    /* Process to which I send each of the vertices */
        &exportToPart);  /* Partition to which each vertex will belong */

  if (rc != ZOLTAN_OK){
    printf("sorry...\n");
    MPI_Finalize();
    Zoltan_Destroy(&zz);
    exit(0);
  }

  /******************************************************************
  ** In a real application, you would rebalance the problem now by
  ** sending the objects to their new partitions.
  ******************************************************************/

\end{verbatim}
\end{flushleft}

\clearpage
\begin{flushleft}
\begin{verbatim}

  /******************************************************************
  ** Free the arrays allocated by Zoltan_LB_Partition, and free
  ** the storage allocated for the Zoltan structure.
  ******************************************************************/

  Zoltan_LB_Free_Part(&importGlobalGids, &importLocalGids, 
                      &importProcs, &importToPart);
  Zoltan_LB_Free_Part(&exportGlobalGids, &exportLocalGids, 
                      &exportProcs, &exportToPart);

  Zoltan_Destroy(&zz);

  /**********************
  ** all done ***********
  **********************/

  MPI_Finalize();

  return 0;
}

\end{verbatim}
\end{flushleft}

\clearpage
\subsection{A graph problem}

In this section we provide an example of a parallel C program
that uses the Zoltan library to partition the simple graph that
was pictured in Figure~\ref{fig:mesh25}.  You can compile
and run this example if you have the Zoltan source code.  It is
in the examples directory, in file \textbf{examples/C/simpleGRAPH.c}.

\subsubsection{Application defined query functions}

We must define four query functions in order to use the
particular graph submethod we chose for this example.  See the
User's Guide at
\url{http://www.cs.sandia.gov/Zoltan/ug_html/ug_alg_parmetis.html} for
the requirements of other submethods.

The function prototypes referred to below
are defined in the Zoltan file \textbf{include/zoltan.h}.
The four query functions are:

\begin{itemize}
\item A function of type ZOLTAN\_NUM\_OBJ\_FN (Figure~\ref{fig:NumObj}) which provides the number of objects belonging to this process 
\item A function of type ZOLTAN\_OBJ\_LIST\_FN (Figure~\ref{fig:ObjList}) which supplies the global IDs and optional weights for the objects belonging to the process
\item A function of type ZOLTAN\_NUM\_EDGES\_MULTI\_FN (Figure~\ref{fig:NumEdges}) which provides the number of edges for each object
\item A function of type ZOLTAN\_EDGE\_LIST\_MULTI\_FN (Figure~\ref{fig:EdgeListMulti})
which, for a given object ID, responds with the IDs for
each adjacent object, the process owning that adjacent object, and optionally a weight
for each edge
\end{itemize}

An alternative to defining a single ZOLTAN\_OBJ\_LIST\_FN is to define
a ZOLTAN\_FIRST\_OBJ\_FN and a ZOLTAN\_NEXT\_OBJ\_FN.  The first function
would return the global ID for first object.  Each call to the second function would
return a subsequent global ID.

An alternative to defining a ZOLTAN\_NUM\_EDGES\_MULTI\_FN is to 
define a ZOLTAN\_NUM\_EDGES\_FN, which returns the number of adjacent
objects for a single object ID.

An alternative to defining a ZOLTAN\_EDGE\_LIST\_MULTI\_FN is to 
define a ZOLTAN\_EDGE\_LIST\_FN, which
returns the same information but for a single object.

The variables used in these query functions to describe our simple graph
refer to those defined in \textbf{examples/C/simpleGraph.h} which were shown in
Figure~\ref{fig:simpleGraphDotH}.

\begin{figure}
\begin{flushleft}
\begin{verbatim}
static void get_num_edges_list(void *data, int sizeGID, int sizeLID,
        int num_obj, ZOLTAN_ID_PTR globalID, ZOLTAN_ID_PTR localID,
        int *numEdges, int *ierr)
{
int i, idx;

  if ( (sizeGID != 1) || (sizeLID != 1)){
    *ierr = ZOLTAN_FATAL;
    return;
  }

  for (i=0;  i < num_obj ; i++){
    idx = globalID[i] - 1;
    numEdges[i] = simpleNumEdges[idx];
  }

  *ierr = ZOLTAN_OK;
  return;
}
\end{verbatim}
\end{flushleft}
\caption{A ZOLTAN\_NUM\_EDGES\_MULTI\_FN query function}
\label{fig:NumEdges}
\end{figure}

\begin{figure}
\begin{flushleft}
\begin{verbatim}
static void get_edge_list(void *data, int sizeGID, int sizeLID,
        int num_obj, ZOLTAN_ID_PTR globalID, ZOLTAN_ID_PTR localID,
        int *num_edges, ZOLTAN_ID_PTR nborGID, int *nborProc,
        int wgt_dim, float *ewgts, int *ierr)
{
int i, j, idx;
ZOLTAN_ID_PTR nextID;
int *nextProc;

  if ( (sizeGID != 1) || (sizeLID != 1) || (wgt_dim != 0)){
    *ierr = ZOLTAN_FATAL;
    return;
  }

  nextID = nborGID;
  nextProc = nborProc;

  for (i=0;  i < num_obj ; i++){
    idx = globalID[i] - 1;
    if (num_edges[i] != simpleNumEdges[idx]){
      *ierr = ZOLTAN_FATAL;
      return;
    }
    for (j=0; j<num_edges[i]; j++){
      *nextID++ = simpleEdges[idx][j];
      *nextProc++ = (simpleEdges[idx][j] - 1) % numProcs;
    }
  }

  *ierr = ZOLTAN_OK;

  return;
}
\end{verbatim}
\end{flushleft}
\caption{A ZOLTAN\_EDGE\_LIST\_MULTI\_FN query function}
\label{fig:EdgeListMulti}
\end{figure}

\clearpage

\subsubsection{Calling the Zoltan library}

This code may be found in the 
file \textbf{examples/C/simpleGRAPH.c}.
More information about graph parameters 
may be found in the Zoltan User's Guide at
\url{http://www.cs.sandia.gov/Zoltan/ug_html/ug_alg_parmetis.html}.

\begin{flushleft}
\begin{verbatim}
#include <mpi.h>
#include <stdio.h>
#include <stdlib.h>
#include "zoltan.h"
#include "simpleGraph.h"
#include "simpleQueries.h"

int main(int argc, char *argv[])
{
  int rc, i, ngids, nextIdx;
  float ver;
  struct Zoltan_Struct *zz;
  int changes, numGidEntries, numLidEntries, numImport, numExport;
  ZOLTAN_ID_PTR importGlobalGids, importLocalGids;
  ZOLTAN_ID_PTR exportGlobalGids, exportLocalGids;
  int *importProcs, *importToPart, *exportProcs, *exportToPart;
  float *wgt_list;
  int *gid_flags, *gid_list, *lid_list;

  /******************************************************************
  ** Initialize MPI and Zoltan
  ******************************************************************/

  MPI_Init(&argc, &argv);
  MPI_Comm_rank(MPI_COMM_WORLD, &myRank);
  MPI_Comm_size(MPI_COMM_WORLD, &numProcs);

  rc = Zoltan_Initialize(argc, argv, &ver);

  if (rc != ZOLTAN_OK){
    MPI_Finalize();
    exit(0);
  }
\end{verbatim}
\end{flushleft}

\clearpage
\begin{flushleft}
\begin{verbatim}
  /******************************************************************
  ** Create a Zoltan library structure for this instance of load
  ** balancing.  Set the parameters and query functions that will
  ** govern the library's calculation.  See the Zoltan User's
  ** Guide for the definition of these and many other parameters.
  ******************************************************************/

  zz = Zoltan_Create(MPI_COMM_WORLD);

  /* General parameters */

  Zoltan_Set_Param(zz, "DEBUG_LEVEL", "0");
  Zoltan_Set_Param(zz, "LB_METHOD", "GRAPH");
  Zoltan_Set_Param(zz, "NUM_GID_ENTRIES", "1"); 
  Zoltan_Set_Param(zz, "NUM_LID_ENTRIES", "1");
  Zoltan_Set_Param(zz, "OBJ_WEIGHT_DIM", "1");
  Zoltan_Set_Param(zz, "RETURN_LISTS", "ALL");

  /* Graph parameters */

  Zoltan_Set_Param(zz, "PARMETIS_METHOD", "PARTKWAY"); 
  Zoltan_Set_Param(zz, "PARMETIS_COARSE_ALG", "2");
  Zoltan_Set_Param(zz, "CHECK_GRAPH", "2"); 

  /* Query functions - defined in simpleQueries.h */

  Zoltan_Set_Num_Obj_Fn(zz, get_number_of_objects, NULL);
  Zoltan_Set_Obj_List_Fn(zz, get_object_list, NULL);
  Zoltan_Set_Num_Edges_Multi_Fn(zz, get_num_edges_list, NULL);
  Zoltan_Set_Edge_List_Multi_Fn(zz, get_edge_list, NULL);
\end{verbatim}
\end{flushleft}

\clearpage
\begin{flushleft}
\begin{verbatim}
  /******************************************************************
  ** Zoltan can now partition the simple graph.
  ** In this simple example, we assume the number of partitions is
  ** equal to the number of processes.  Process rank 0 will own
  ** partition 0, process rank 1 will own partition 1, and so on.
  ******************************************************************/

  rc = Zoltan_LB_Partition(zz, /* input (all remaining fields are output) */
        &changes,        /* 1 if partitioning was changed, 0 otherwise */ 
        &numGidEntries,  /* Number of integers used for a global ID */
        &numLidEntries,  /* Number of integers used for a local ID */
        &numImport,      /* Number of vertices to be sent to me */
        &importGlobalGids,  /* Global IDs of vertices to be sent to me */
        &importLocalGids,   /* Local IDs of vertices to be sent to me */
        &importProcs,    /* Process rank for source of each incoming vertex */
        &importToPart,   /* New partition for each incoming vertex */
        &numExport,      /* Number of vertices I must send to other processes*/
        &exportGlobalGids,  /* Global IDs of the vertices I must send */
        &exportLocalGids,   /* Local IDs of the vertices I must send */
        &exportProcs,    /* Process to which I send each of the vertices */
        &exportToPart);  /* Partition to which each vertex will belong */

  if (rc != ZOLTAN_OK){
    printf("sorry...\n");
    MPI_Finalize();
    Zoltan_Destroy(&zz);
    exit(0);
  }

  /******************************************************************
  ** In a real application, you would rebalance the problem now by
  ** sending the objects to their new partitions.  Your query 
  ** functions would need to reflect the new partitioning.
  ******************************************************************/
\end{verbatim}
\end{flushleft}

\clearpage
\begin{flushleft}
\begin{verbatim}
  /******************************************************************
  ** Free the arrays allocated by Zoltan_LB_Partition, and free
  ** the storage allocated for the Zoltan structure.
  ******************************************************************/

  Zoltan_LB_Free_Part(&importGlobalGids, &importLocalGids, 
                      &importProcs, &importToPart);
  Zoltan_LB_Free_Part(&exportGlobalGids, &exportLocalGids, 
                      &exportProcs, &exportToPart);

  Zoltan_Destroy(&zz);

  /**********************
  ** all done ***********
  **********************/

  MPI_Finalize();

  return 0;
}
\end{verbatim}
\end{flushleft}

\clearpage
\subsection{A hypergraph problem}

In this example, we use the simple graph in  Figure~\ref{fig:mesh25}
to create a hypergraph, and
then partition it using Zoltan's parallel hypergraph
partitioner.  

In a hypergraph, each edge can be associated with more than
two vertices.  This edge is called a hyperedge.

For each vertex in the simple graph, we create a hyperedge
with a vertex list composed of that vertex and all of its
graph neighbors.  So our hypergraph has 25 hyperedges, one
for each vertex in the simple graph.

We are creating this hypergraph in the hypergraph query functions.
But you can use the Zoltan parameter
\emph{PHG\_FROM\_GRAPH\_METHOD=neighbors} 
described in the User's Guide at 
\url{http://www.cs.sandia.gov/Zoltan/ug_html/ug_alg_phg.html},
to accomplish the same
result.  In this case you use graph query functions instead of
hypergraph query funtions to supply the 
graph to Zoltan.
The Zoltan library would then do the conversion to a hypergraph.

First we will define the required query functions.

\subsubsection{Application defined query functions}

We must define six query functions.  The function prototyes named 
below are defined in the Zoltan file \textbf{include/zoltan.h}.

\begin{itemize}
\item A function of type ZOLTAN\_NUM\_OBJ\_FN (Figure~\ref{fig:NumObj}) which provides the number of objects (vertices) belonging to this process .
\item A function of type ZOLTAN\_OBJ\_LIST\_FN (Figure~\ref{fig:ObjList2}) which supplies the object global IDs and optional weights.  It differs from the query function created for the last two examples (Figure~\ref{fig:ObjList}) because it omits the optional local IDs and weights.
\item A function of type ZOLTAN\_HG\_SIZE\_CS\_FN (Figure~\ref{fig:SizeCS}).  Each process will supply a portion of the hypergraph in a compressed storage format.  This function tells the Zoltan library the size of the information that will be returned by the application process.
\item A function of type ZOLTAN\_HG\_CS\_FN (Figure~\ref{fig:CS}) which supplies a portion of the hypergraph in a compressed storage format.
\item A function of type ZOLTAN\_HG\_SIZE\_EDGE\_WTS\_FN (Figure~\ref{fig:SizeEW}) which supplies the number of hyperedges for which the process will provide edge weights.
\item A function of type ZOLTAN\_HG\_EDGE\_WTS\_FN (Figure~\ref{fig:EW}) which supplies optional edge weights for the hyperedges.
\end{itemize}

An alternative to defining a single ZOLTAN\_OBJ\_LIST\_FN is to define
a ZOLTAN\_FIRST\_OBJ\_FN and a ZOLTAN\_NEXT\_OBJ\_FN.  The first function
would return the global ID for first object.  Each call to the second function would
return a subsequent global ID.

In this example the application is maintaining a small structure
with information about the size of the hypergraph.  The application
supplies the address of this structure to the Zoltan library
when setting up the query functions.  Zoltan supplies the address
of the structure to the application when calling the query functions.

If you have the Zoltan source, the following query functions
are defined in the file \textbf{examples/C/simpleQueries.h}.
The variables used in these query functions refer to those defined 
in \textbf{examples/C/simpleGraph.h} which were shown in
Figure~\ref{fig:simpleGraphDotH}.

\begin{figure}
\begin{flushleft}
\begin{verbatim}
static void get_hg_object_list(void *data, int sizeGID, int sizeLID,
            ZOLTAN_ID_PTR globalID, ZOLTAN_ID_PTR localID,
            int wgt_dim, float *obj_wgts, int *ierr)
{
int i, next;

  if ((sizeGID != 1) || (sizeLID != 0) || (wgt_dim != 0)) {
    *ierr = ZOLTAN_FATAL;
    return;
  }

  for (i=0, next=0; i<simpleNumVertices; i++){
    if (i % numProcs == myRank){
      globalID[next] = i+1;   /* application wide global ID */
      next++;
    }
  }

  *ierr = ZOLTAN_OK;

  return;
}
\end{verbatim}
\end{flushleft}
\caption{A ZOLTAN\_OBJ\_LIST\_FN query function}
\label{fig:ObjList2}
\end{figure}

\begin{figure}
\begin{flushleft}
\begin{verbatim}
static struct _hg_data{
  int numEdges;
  int numPins;
}hg_data;

void get_hg_size(void *data, int *num_lists, int *num_pins,
                 int *format, int *ierr)
{
int i;
struct _hg_data *hgd;

  hgd = (struct _hg_data *)data;

  hgd->numEdges = 0;
  hgd->numPins  = 0;

  for (i=0; i<simpleNumVertices; i++){
    if (i % numProcs == myRank){
      hgd->numPins += simpleNumEdges[i] + 1;  /* my hyperedge */
      hgd->numEdges++;
    }
  }

  *num_lists = hgd->numEdges;
  *num_pins  = hgd->numPins;
  *format    = ZOLTAN_COMPRESSED_EDGE;
  *ierr      = ZOLTAN_OK;

  return;
}
\end{verbatim}
\end{flushleft}
\caption{A ZOLTAN\_HG\_SIZE\_CS\_FN query function}
\label{fig:SizeCS}
\end{figure}

\begin{figure}
\begin{flushleft}
\begin{verbatim}
void get_hg(void *data, int sizeGID, int num_rows, int num_pins,
            int format, ZOLTAN_ID_PTR edge_GID, int *edge_ptr,
            ZOLTAN_ID_PTR pin_GID, int *ierr)
{
int i, j, npins;
struct _hg_data *hgd;

  hgd    = (struct _hg_data *)data;

  if ((num_rows != hgd->numEdges) || (num_pins != hgd->numPins) ||
      (format != ZOLTAN_COMPRESSED_EDGE)){
    *ierr = ZOLTAN_FATAL;
    return;
  }

  npins = 0;

  for (i=0; i<simpleNumVertices; i++){
    if (i % numProcs == myRank){    /* my hyperedge */
      *edge_ptr++ = npins;      /* index into start of pin list */
      *edge_GID++ = i+1;        /* hyperedge global ID */

      /* list global ID of each pin (vertex) in hyperedge */
      for (j=0; j<simpleNumEdges[i]; j++){
        *pin_GID++ = simpleEdges[i][j];
      }
      *pin_GID++ = i+1;
      npins += simpleNumEdges[i] + 1;
    }
  }

  *ierr = ZOLTAN_OK;

  return;
}
\end{verbatim}
\end{flushleft}
\caption{A ZOLTAN\_HG\_CS\_FN query function}
\label{fig:CS}
\end{figure}

\begin{figure}
\begin{flushleft}
\begin{verbatim}
void get_hg_num_edge_weights(void *data, int *num_edges, int *ierr)
{
struct _hg_data *hgd;

  hgd    = (struct _hg_data *)data;

  *num_edges = hgd->numEdges;
  *ierr = ZOLTAN_OK;

  return;
}
\end{verbatim}
\end{flushleft}
\caption{A ZOLTAN\_HG\_SIZE\_EDGE\_WTS\_FN query function}
\label{fig:SizeEW}
\end{figure}

\begin{figure}
\begin{flushleft}
\begin{verbatim}
void get_hyperedge_weights(void *data, int sizeGID,
         int sizeLID, int num_edges, int edge_weight_dim,
         ZOLTAN_ID_PTR edge_GID, ZOLTAN_ID_PTR edge_LID,
         float  *edge_weight, int *ierr)
{
int i;
struct _hg_data *hgd;

  hgd    = (struct _hg_data *)data;

  if ((sizeGID != 1) || (sizeLID != 0) ||
      (num_edges != hgd->numEdges) || (edge_weight_dim != 1)){
    *ierr = ZOLTAN_FATAL;
    return;
  }

  for (i=0; i<simpleNumVertices; i++){
    if (i % numProcs == myRank){
      *edge_GID++ = i+1;        /* hyperedge global ID */
      *edge_weight++ = 1.0;  /* all hyperedges same weight */
    }
  }
  *ierr = ZOLTAN_OK;
  return;
}
\end{verbatim}
\end{flushleft}
\caption{A ZOLTAN\_HG\_EDGE\_WTS\_FN query function}
\label{fig:EW}
\end{figure}

\clearpage
\subsubsection{Calling the Zoltan library}
If you have the Zoltan source, the file containing this
code is \textbf{examples/C/simpleHG.c}.
More information about the hypergraph parameters
may be found in the Zoltan User's Guide at
\url{http://www.cs.sandia.gov/Zoltan/ug_html/ug_alg_phg.html}.

\begin{flushleft}
\begin{verbatim}
#include <mpi.h>
#include <stdio.h>
#include <stdlib.h>
#include "zoltan.h"
#include "simpleGraph.h"
#include "simpleQueries.h"

int main(int argc, char *argv[])
{
  int rc, i, ngids, nextIdx;
  float ver;
  struct Zoltan_Struct *zz;
  int changes, numGidEntries, numLidEntries, numImport, numExport;
  ZOLTAN_ID_PTR importGlobalGids, importLocalGids;
  ZOLTAN_ID_PTR exportGlobalGids, exportLocalGids;
  int *importProcs, *importToPart, *exportProcs, *exportToPart;
  int *gid_flags, *gid_list;

  /******************************************************************
  ** Initialize MPI and Zoltan
  ******************************************************************/

  MPI_Init(&argc, &argv);
  MPI_Comm_rank(MPI_COMM_WORLD, &myRank);
  MPI_Comm_size(MPI_COMM_WORLD, &numProcs);

  rc = Zoltan_Initialize(argc, argv, &ver);

  if (rc != ZOLTAN_OK){
    MPI_Finalize();
    exit(0);
  }
\end{verbatim}
\end{flushleft}

\clearpage
\begin{flushleft}
\begin{verbatim}
  /******************************************************************
  ** Create a Zoltan library structure for this instance of load
  ** balancing.  Set the parameters and query functions that will
  ** govern the library's calculation.  See the Zoltan User's
  ** Guide for the definition of these and many other parameters.
  ******************************************************************/

  zz = Zoltan_Create(MPI_COMM_WORLD);

  /* General parameters */

  Zoltan_Set_Param(zz, "DEBUG_LEVEL", "0");
  Zoltan_Set_Param(zz, "LB_METHOD", "HYPERGRAPH");
  Zoltan_Set_Param(zz, "NUM_GID_ENTRIES", "1"); 
  Zoltan_Set_Param(zz, "NUM_LID_ENTRIES", "0");
  Zoltan_Set_Param(zz, "OBJ_WEIGHT_DIM", "0");
  Zoltan_Set_Param(zz, "RETURN_LISTS", "ALL");

  /* Hypergraph parameters */

  Zoltan_Set_Param(zz, "HYPERGRAPH_PACKAGE", "PHG"); 
  Zoltan_Set_Param(zz, "PHG_COARSENING_METHOD", "ipm");
  Zoltan_Set_Param(zz, "PHG_COARSEPARTITION_METHOD", "greedy");
  Zoltan_Set_Param(zz, "PHG_REFINEMENT_METHOD", "fm");
  Zoltan_Set_Param(zz, "PHG_EDGE_WEIGHT_OPERATION", "error"); 
  Zoltan_Set_Param(zz, "PHG_EDGE_SIZE_THRESHOLD", ".7"); 
  Zoltan_Set_Param(zz, "ADD_OBJ_WEIGHT", "unit"); 

  /* Query functions - defined in simpleQueries.h */

  Zoltan_Set_Num_Obj_Fn(zz, get_number_of_objects, &hg_data);
  Zoltan_Set_Obj_List_Fn(zz, get_hg_object_list, &hg_data);
  Zoltan_Set_HG_Size_CS_Fn(zz, get_hg_size, &hg_data);
  Zoltan_Set_HG_CS_Fn(zz, get_hg, &hg_data);
  Zoltan_Set_HG_Size_Edge_Wts_Fn(zz, get_hg_num_edge_weights, &hg_data);
  Zoltan_Set_HG_Edge_Wts_Fn(zz, get_hyperedge_weights, &hg_data);
\end{verbatim}
\end{flushleft}

\clearpage
\begin{flushleft}
\begin{verbatim}
  /******************************************************************
  ** Zoltan can now partition the vertices.
  ** In this simple example, we assume the number of partitions is
  ** equal to the number of processes.  Process rank 0 will own
  ** partition 0, process rank 1 will own partition 1, and so on.
  ******************************************************************/

  rc = Zoltan_LB_Partition(zz, /* input (all remaining fields are output) */
        &changes,        /* 1 if partitioning was changed, 0 otherwise */ 
        &numGidEntries,  /* Number of integers used for a global ID */
        &numLidEntries,  /* Number of integers used for a local ID */
        &numImport,      /* Number of vertices to be sent to me */
        &importGlobalGids,  /* Global IDs of vertices to be sent to me */
        &importLocalGids,   /* Local IDs of vertices to be sent to me */
        &importProcs,    /* Process rank for source of each incoming vertex */
        &importToPart,   /* New partition for each incoming vertex */
        &numExport,      /* Number of vertices I must send to other processes*/
        &exportGlobalGids,  /* Global IDs of the vertices I must send */
        &exportLocalGids,   /* Local IDs of the vertices I must send */
        &exportProcs,    /* Process to which I send each of the vertices */
        &exportToPart);  /* Partition to which each vertex will belong */

  if (rc != ZOLTAN_OK){
    printf("sorry...\n");
    MPI_Finalize();
    Zoltan_Destroy(&zz);
    exit(0);
  }

  /******************************************************************
  ** In a real application, you would rebalance the problem now by
  ** sending the objects to their new partitions.  Your query 
  ** functions would need to reflect the new partitioning.
  ******************************************************************/
\end{verbatim}
\end{flushleft}

\clearpage
\begin{flushleft}
\begin{verbatim}
  /******************************************************************
  ** Free the arrays allocated by Zoltan_LB_Partition, and free
  ** the storage allocated for the Zoltan structure.
  ******************************************************************/

  Zoltan_LB_Free_Part(&importGlobalGids, &importLocalGids, 
                      &importProcs, &importToPart);
  Zoltan_LB_Free_Part(&exportGlobalGids, &exportLocalGids, 
                      &exportProcs, &exportToPart);

  Zoltan_Destroy(&zz);

  /**********************
  ** all done ***********
  **********************/

  MPI_Finalize();

  return 0;
}
\end{verbatim}
\end{flushleft}

\clearpage
\section{A C++ example}

In this section we show a C++ example that solves the same
recursive coordinate bisection problem that the C example
in Section~\ref{sec:rcb} solves.

We created a C++ class called \textbf{rectangularMesh}, which can represent
the rectangular mesh shown in Figure~\ref{fig:mesh25}.
The query functions required by Zoltan are implemented as 
static methods of the \textbf{rectangularMesh} class.  They
take as an argument an object of type \textbf{rectangularMesh}.
We will not list the entire class due to space constraints.  If you
obtain the Zoltan source you will find the class in
\textbf{examples/CPP/rectangularMesh.h}.  But just to give you an
idea, here is the method that returns the number of objects
owned by the process:

\begin{flushleft}
\begin{verbatim}
  static int get_number_of_objects(void *data, int *ierr)
    {
    // Prototype: ZOLTAN_NUM_OBJ_FN
    // Return the number of objects I own.
    
    rectangularMesh *mesh = (rectangularMesh *)data;
    *ierr = ZOLTAN_OK;
    return mesh->get_num_my_IDs();
    }
\end{verbatim}
\end{flushleft}

We use static class methods, but your
query functions could also be global C or C++ functions.

The example below can be found in the Zoltan source in 
\textbf{examples/CPP/simpleRCB.cpp}.

\clearpage
\begin{flushleft}
\begin{verbatim}

#include <rectangularGraph.h>
int main(int argc, char *argv[])
{
  // Initialize MPI and Zoltan

  int rank, size;
  float version;

  MPI::Init(argc, argv);
  rank = MPI::COMM_WORLD.Get_rank();
  size = MPI::COMM_WORLD.Get_size();

  Zoltan_Initialize(argc, argv, &version);

  // Create a Zoltan object 
  
  Zoltan *zz = new Zoltan(MPI::COMM_WORLD);

  if (zz == NULL){
    MPI::Finalize();
    exit(0);
  }

  // Create a simple rectangular mesh.  It's vertices are the
  // objects to be partitioned.

  rectangularMesh *mesh = new rectangularMesh();

  mesh->set_x_dim(10);
  mesh->set_y_dim(8);
  mesh->set_x_stride(1);
  mesh->set_y_stride(4);

  // General parameters:

  zz->Set_Param("DEBUG_LEVEL", "0");     // amount of debug messages desired
  zz->Set_Param("LB_METHOD", "RCB");     // recursive coordinate bisection
  zz->Set_Param("NUM_GID_ENTRIES", "1"); // number of integers in a global ID
  zz->Set_Param("NUM_LID_ENTRIES", "1"); // number of integers in a local ID
  zz->Set_Param("OBJ_WEIGHT_DIM", "1");  // dimension of a vertex weight
  zz->Set_Param("RETURN_LISTS", "ALL");  // return all lists in LB_Partition

  // RCB parameters:

  zz->Set_Param("KEEP_CUTS", "1");              // save decomposition
  zz->Set_Param("RCB_OUTPUT_LEVEL", "0");       // amount of output desired
  zz->Set_Param("RCB_RECTILINEAR_BLOCKS", "1"); // create rectilinear regions

  // Query functions:

  zz->Set_Num_Obj_Fn(rectangularMesh::get_number_of_objects, (void *)mesh);
  zz->Set_Obj_List_Fn(rectangularMesh::get_object_list, (void *)mesh);
  zz->Set_Num_Geom_Fn(rectangularMesh::get_num_geometry, NULL);
  zz->Set_Geom_Multi_Fn(rectangularMesh::get_geometry_list, (void *)mesh);
  
  // Zoltan can now partition the vertices in the simple mesh.

  int changes;
  int numGidEntries;
  int numLidEntries;
  int numImport;
  ZOLTAN_ID_PTR importGlobalIds;
  ZOLTAN_ID_PTR importLocalIds;
  int *importProcs;
  int *importToPart;
  int numExport;
  ZOLTAN_ID_PTR exportGlobalIds;
  ZOLTAN_ID_PTR exportLocalIds;
  int *exportProcs;
  int *exportToPart;

  int rc = zz->LB_Partition(changes, numGidEntries, numLidEntries,
    numImport, importGlobalIds, importLocalIds, importProcs, importToPart,
    numExport, exportGlobalIds, exportLocalIds, exportProcs, exportToPart);
  
  if (rc != ZOLTAN_OK){
    printf("Partitioning failed on process %d\n",rank);
    MPI::Finalize();
    delete zz;
    delete mesh;
    exit(0);
  }

  // In a real application, you would rebalance the problem now by
  // sending the objects to their new partitions.

  // Free the arrays allocated by LB_Partition, and free
  // the storage allocated for the Zoltan structure and the mesh.

  zz->LB_Free_Part(&importGlobalIds, &importLocalIds, &importProcs,
                   &importToPart);
  zz->LB_Free_Part(&exportGlobalIds, &exportLocalIds, &exportProcs,
                   &exportToPart);

  delete zz;
  delete mesh;

  // all done ////////////////////////////////////////////////////

  MPI::Finalize();

  return 0;
}
\end{verbatim}
\end{flushleft}

