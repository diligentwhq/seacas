\chapter{Operators}\label{ch:operators}

The operators recognized by \aprepro{} are listed below. The letters \textbf{a} 
and \textbf{b} can represent variables, numbers, functions, or expressions unless 
otherwise noted. The tables below also list the precedence and associativity of 
the operators. {\em Precedence} defines the order in which operations should be performed. 
For example, in the expression:
\begin{apinp}
\{3 * 4 + 6 / 2\}
\end{apinp}
the multiplications and divisions are performed first, followed by the addition 
because multiplication and division have higher precedence (10) than addition (9). The precedence 
is listed from 1 to 14 with 1 being the lowest precedence and 14 being the highest.

{\em Associativity} defines which side of the expressions should be
simplified first.  For example the expression: \cmd{3 + 4 + 5} would
be evaluated as \cmd{(3 + 4) + 5} since addition is left associative;
in the expression \textbf{a = b / c}, the \cmd{b/c} would be evaluated
first followed by the assignment of that result to \cmd{a} since
equality is right associative

\section{Arithmetic Operators}

Arithmetic operators combine two or more algebraic expressions into a
single algebraic expression. These have obvious meanings except for
the pre- and post- increment and decrement operators. The
pre-increment and pre-decrement operators first increment or decrement
the value of the variable and then return the value. For example, if
\cmd{a = 1}, then \cmd{b=++a} will set both \cmd{b} and \cmd{a} equal to \cmd{2}. The post-increment and
post-decrement operators first return the value of the
variable and then increment or decrement the variable. For example, if
\cmd{a = 1}, then \cmd{b=a++} will set
\cmd{b} equal to \cmd{1} and \cmd{a} equal to \cmd{2}. The
modulus operator \cmd{\%} calculates the integer remainder. That is
both expressions are truncated an integer value and then the remainder
calculated. See the \cmd{fmod} function in Table~\ref{t:functions} for
the calculation of the floating point remainder. The tilde character
\cmd{\textasciitilde{}} is used as a synonym for multiplication to
improve the aesthetics of the unit conversion system (see
Chapter~\ref{ch:units}). It is more natural for some users to type
\cmd{12\textasciitilde{}metre} than \cmd{12*metre}.


\begin{longtable}{llcc}
\caption{Arithmetic Operators}\\
Syntax   & Description    & Precedence & Associativity\\
\hline
\cmd{a+b}      & Addition       &  9 & left \\
\cmd{a-b}      & Subtraction    &  9 & left \\
\cmd{a*b}, \cmd{a}\verb+~+\cmd{b} & Multiplication & 10 & left \\
\cmd{a/b}      & Division       & 10 & left \\
\cmd{a\^{}b}, \cmd{a**b} & Exponentiation & 12 & right \\
\cmd{a\%b}      & Modulus, (remainder) & 10 & left \\
\cmd{++a}, \cmd{a++} & Pre- and Post-increment a & 13 & left\\
\cmd{\texttt{--}a}, \cmd{a\texttt{--}} & Pre- and Post-decrement a & 13 & left\\
\end{longtable}


\section{Assignment Operators}

Assignment operators combine a variable and an algebraic expression into a single 
algebraic expression, and also set the variable equal to the algebraic expression. 
Only variables can be specified on the left-hand-side of the equal sign.

\begin{longtable}{llcc}
\caption{Assignment Operators}\\
Syntax & Description & Precedence & Associativity \\
\hline
\cmd{a=b}   & The value of $a$ is set equal to $b$ & 1 & right\\
\cmd{a+=b}  & The value of $a$ is set equal to $a + b$ & 2 & right\\
\cmd{a-=b}  & The value of $a$ is set equal to $a - b$ & 2 & right\\
\cmd{a*=b}  & The value of $a$ is set equal to $a * b$ & 3 & right\\
\cmd{a/=b}  & The value of $a$ is set equal to $a / b$ & 3 & right\\
\cmd{a}\^{}\cmd{=b}, a\cmd{**=b}& The value of $a$ is set equal to  $a^b$ & 4 & right\\
\end{longtable}

\section{Relational Operators}\label{sec:relationaloperators}

Relational operators combine two algebraic expressions into a single relational 
expression. Relational expressions and operators can only be used before the question 
mark (\cmd{?}) in a conditional expression.

\begin{longtable}{llcc}
\caption{Relational Operators}\\
Syntax & Description & Precedence & Associativity \\
\hline
\cmd{a \texttt{<} b} & true if $a$ is less than $b$ & 8 & left\\
\cmd{a \texttt{>} b} & true if $a$ is greater than $b$ & 8 & left\\
\cmd{a \texttt{<}= b} & true if $a$ is less than or equal to $b$ & 8 & left\\
\cmd{a \texttt{>}= b} & true if $a$ is greater than or equal to $b$ & 8 & left\\
\cmd{a ==  b} & true if $a$ is equal to $b$ & 8 & left\\
\cmd{a !=  b} & true if $a$ is not equal to $b$ & 8 & left\\
\end{longtable}

\section{Boolean Operators}

Boolean operators combine one or more relational expressions into a single relational 
expression. If \cmd{la} and \cmd{lb} are two relational expressions, then:

\begin{longtable}{llcc}
\caption{Boolean Operators}\\
Syntax & Description & Precedence & Associativity \\
\hline
\cmd{la \textbar{}\textbar{} lb} & true if either $la$ or $lb$ are true. & 6 & left\\
\cmd{la \&\&  lb} & true if both $la$ and $lb$ are true. & 7 & left\\
\cmd{!la} & true if $la$ is false. & 11 & left\\
\end{longtable}
The evaluation of the expression is not short-circuited if the truth
value can be determined early; both sides of the expression are
evaluated and then the truth of the expression is returned.

\section{String Operators}

The only supported string operator at this time is string concatenation which is 
denoted by \cmd{//}. For example,
\begin{apinp}
\{a = "Hello"\} \{b = "World"\}
\{c = a // " " //  b\}
\end{apinp}
sets \cmd{c} equal to \cmd{\texttt{"}Hello  World\texttt{"}.}  Concatenation 
has precedence 14 and left associativity.
