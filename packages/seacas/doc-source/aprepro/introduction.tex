\chapter{Introduction}

\aprepro{} is an algebraic preprocessor that reads a file containing both 
general text and algebraic expressions. It echoes the general text to the output 
file, along with the results of the algebraic expressions. The syntax used in \aprepro{}
is such that all expressions between the delimiters \textbf{\{} and \textbf{\}} 
are evaluated and all other text is simply echoed to the output file. For example, 
if the following lines are input to \aprepro{}:

\begin{apinp}
$ Rad  = \{Rad  =  12.0\}
Point 1 \{x1 = Rad * sind(30.)\} \{y1 = Rad * cosd(30.)\}
Point 2 \{x1 + 10.0\}            \{y1\}
\end{apinp}

The output would look like:

\begin{apout}
$ Rad = 12
Point  1 6   10.39230485
Point  2 16  10.39230485
\end{apout}

In this example, the algebraic expressions are specified by surrounding them with 
\textbf{\{} and \textbf{\}}, and the functions \afunc{sind()} and \afunc{cosd()} 
calculate the sine and cosine of an angle given in degrees.

\aprepro{} has been used extensively for several years to prepare parameterized 
files for finite element analyses using the Sandia National Laboratories 
SEACAS system~\cite{bib:seacas}. The units conversion capability has greatly increased the 
usability of \aprepro{}. \aprepro{} can also be used for non-finite element 
applications such as a powerful calculator and a general text processor for any 
file that does not use the delimiters \textbf{\{} and \textbf{\}}.

The remainder of this document is organized as follows:

\begin{itemize}
\item Chapter~\ref{ch:execution} documents the command line options for \aprepro{} and the text input, editing, and recall capabilities.
\item Chapter~\ref{ch:syntax} documents the syntax recognized by \textit{Aprepro}, 
\item Chapters \ref{ch:operators}, \ref{ch:predefined}, and~\ref{ch:functions} describe the operators, predefined variables, and functions, 
\item Chapter~\ref{ch:units} describes the units conversion system, 
\item Chapter~\ref{ch:errors} describes the error messages output from \aprepro{}, and 
\item Chapter~\ref{ch:examples} presents some examples of \aprepro{} usage.
\end{itemize}
