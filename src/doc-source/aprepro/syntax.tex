\chapter{Syntax}\label{ch:syntax}

\aprepro{} is in one of two states while it is processing an input file, 
either echoing or parsing. In the \textit{echoing} state, \aprepro{}
echoes every character that it reads to the output file. If it reads
the character \cmd{\{}, it enters the \textit{parsing} state. In
the parsing state, \aprepro{} reads characters from the input file
and identifies the characters as tokens which can be \textit{function
names}, \textit{variables}, \textit{numbers}, \textit{operators}, or
\textit{delimiters}.  When \aprepro{} encounters the character
\cmd{\}}, it tries to interpret the tokens as an algebraic, string,
or conditional expression; if it is successful, it prints the value
to the output file; if it cannot evaluate the expression, it prints
the message:

\cmd{Aprepro: ERROR: parse  error \{filename\},  line \{line\#\}}

to the terminal\footnote{Error messages are printed to standard
error. On UNIX systems they can be redirected to a file using your
shells redirection syntax. See the man page for your shell for more
information.} prints the value 0 to the output file.

The rules that \aprepro{} uses when identifying functions, variables, numbers, 
operators, delimiters, and expressions are described below:

\begin{description}

\item[Functions] Function names are sequences of letters and digits and underscores
\cmd{(\_)} that begin with a letter. The function's arguments are
enclosed in parentheses.

For example, in the line \cmd{atan2(a,1.0)}, \afunc{atan2} is the
function name, and \cmd{a} and \cmd{1.0 }are the arguments. See
Chapter~\ref{ch:functions} for a list of the available functions and
their arguments.

\item[Variables] A variable is a name that references a numeric or string 
value. A variable is defined by giving it a name and assigning it a value. For 
example, the expression \cmd{a = 1.0} defines the variable \cmd{a} with 
the numeric value \cmd{1.0}; the expression \cmd{b= \texttt{"}A string\texttt{"}} 
defines the variable \cmd{b} with the value \cmd{A string}. 
Variable names are sequences of letters, digits, colons (:), and underscores (\_) that begin 
with either a letter or an underscore. Variable names cannot match any function 
name and they are case-sensitive, that is, \cmd{abc\_de} and \cmd{AbC\_dE} 
are two distinct variable names. A few variables are predefined, these are listed 
in  Chapter~\ref{ch:predefined}.

Any variable that is not defined is equal to 0. A warning message is
output to the terminal if an undefined variable is used, or if a
previously defined variable is redefined. If the variable
name begins with an underscore, no warning is output when the variable
is redefined.\footnote{Warnings can be turned off with the \cmd{-W} or
\cmd{--warning} option.}.

\item[Immutable Variables]\label{immutable} An immutable variable is a variable whose
value cannot be changed. An immutable variable name is follows the
same rules as a regular variable except that the name cannot begin
with an underscore.  Immutable variables are created inside an
\cmd{IMMUTABLE(ON)} block (See Section~\ref{sec:immutable}) or when
\aprepro{} is executed with the \cmd{--immutable} or \cmd{-X} command
line options (See Chapter~\ref{ch:execution}). If the value of an
immutable variable is attempted to be modified, an error message of
the form:
\cmd[Aprepro: (IMMUTABLE) Variable 'variable' is immutable and cannot be modified (file, line line\#)]
will be output to the standard error stream and the expression
containing the assignment to the immutable variable will return nothing.

\item[Numbers] Numbers can be integers like \cmd{1234}, decimal numbers 
like \cmd{1.234}, or in scientific notation like \cmd{1.234E-26}. All numbers 
are stored internally as floating point numbers.

\item[Strings] Strings are sequences of numbers, characters, and symbols 
that are delimited by either single quotes ('\cmd{this is a
string'}) or double quotes (\cmd{\texttt{"}this} \cmd{is another
string\texttt{"}}). Strings that are delimited by one type of quote
can include the other type of quote. For example, \cmd{'This is a
valid \texttt{"}string\texttt{"}'}.  Strings delimited by single
quotes can span multiple lines; strings delimited by double quotes
must terminate on a single line or a parsing error message will be
issued.

\item[Operators] 
Operators are any of the symbols defined in
Chapter~\ref{ch:operators}. Examples are + (addition), -
(subtraction), * (multiplication), / (division), = (assignment), and
\textasciicircum{} (exponentiation)

\item[Delimiters] The delimiters recognized by \aprepro{} are: the 
comma (\cmd{,}) which separates arguments in function lists, the
left curly brace (\cmd{\{}) which begins an expression, the right
curly brace (\cmd{\}}) which ends an expression, the left parenthesis
\cmd{(} which begins a function argument list, the right
parenthesis \cmd{)} which ends a function argument list, the
single quote (\cmd{'}) which delimits a multiline string, and the double
quote (\cmd{\texttt{"}}) which delimits a single-line string. If a left
or right curly brace is needed in the processes output without being
interpreted by \aprepro{}, precede the curly brace with a backslash.
For example, \verb+\+\cmd{\{ }\verb+\+\cmd{\}}.

\item[Expressions] An expression consists of any combination of numeric 
and string constants, variables, operators, and functions. Four types
of expressions are recognized in \aprepro{}: algebraic, string,
relational, and conditional.

\item[Algebraic Expressions] Almost any valid FORTRAN or C algebraic expression 
can be recognized and evaluated by \aprepro{}. An expression of the
form
\cmd{a=b+10/37.5} will evaluate the expression on the right-hand-side 
of the equals sign, print the value to the output file, and assign the
value to the variable \cmd{a}. An expression of the form
\cmd{b+10/37.5} will evaluate the expression and print the value to
the output file. If you want to assign a value to a variable without
printing the result, the expression must be inside an
\cmd{ECHO(ON\textbar{}OFF)} block (see~\pageref{echo}). Variables can
also be set on the command line prior to reading any input files using
the \cmd{var=val} syntax. An example of this usage is given in
Section~\ref{example:varval}. Only a single expression is allowed
within the \cmd{\{ \}} delimiters. For example,
\cmd{\{x=sqrt(y\textasciicircum{}2 + sin(z))\}}, \cmd{\{x=y=z\}},
and\cmd{\{x=y\} \{a=z\}} are valid expressions, but \cmd{\{x=y a=z\}}
is invalid because it contains two expressions within a single set of
delimiters.

\item[String Expressions] \aprepro{} has limited string support. 
The only supported operations are (1)~assigning a variable equal to a
string (\cmd{a = \texttt{"}This is a string\texttt{"}}), (2)~functions
that return a string (See Table~\ref{t:stringfunctions}), and (3)~concatenating two strings into another
string (\cmd{a = \texttt{"}Hello\texttt{"} // \texttt{"}
\texttt{"} // \texttt{"}World\texttt{"}}).

\item[Relational Expressions:] Relational expressions are expressions that 
return the result of comparing two expressions. A relational
expression is either true (returns 1) or false (returns 0). A
relational expression is simply two expressions of any kind separated
by a relational operator (See Section~\ref{sec:relationaloperators}).

\item[Conditional Expressions] \aprepro{} recognizes a conditional 
expression of the form:
\begin{apinp}
relational\_expression ? true\_exp :  false\_exp
\end{apinp}
where \cmd{relational\_expression} can be any valid
relational expression, and \cmd{true\_exp} and
\cmd{false\_exp} are two algebraic expressions or string expressions. If the
relational expression is true, then the result of
\cmd{true\_exp} is returned, otherwise the result of
\cmd{false\_exp} is returned. For example, if the
following command were entered:
\begin{apinp}
a = (sind(20.0) > cosd(20.0) ? 1 : -1)
\end{apinp}
then, \cmd{a} would be assigned the value \cmd{-1} since the relational expression 
to the left of the question mark is false. Both \textit{\cmd{true\_exp}} and 
\textit{\cmd{false\_exp}} are always evaluated prior to evaluating the relational 
expression. Therefore, you should not write an equation such as
\begin{apinp}
sind(20.0*a) > cosd(20.0*a) ? a=sind(20.0) : a=cosd(20.0)
\end{apinp}
since the value of \cmd{a} can change during the evaluation of the expression. 
Instead, this equation should be written as:
\begin{apinp}
a = (sind(20.0*a) > cosd(20.0*a) ? sind(20.0) : cosd(20.0))
\end{apinp}
\end{description}
