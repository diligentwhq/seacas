\chapter{Units Conversion System}\label{ch:units}

\section{Introduction}

The units conversion system as implemented in \aprepro{} defines
several variables that are abbreviations for unit quantities. For
example, if the output format for the current unit system was inches,
the variable \cmd{foot} would have the value \cmd{12}. Therefore, an
expression such as \cmd{8*foot} would be equal to \cmd{96} which is the
number of inches in 8 feet\footnote{This can also be written as 8\~{}foot since \~{} has been defined to be the same as $*$ (multiplication).}.

Seven consistent units systems have been defined including four metric
based systems: \cmd{si}, \cmd{cgs}, \cmd{cgs-ev}, and \cmd{shock}; and
three english-based systems: \cmd{in-lbf-s}, \cmd{ft-lbf-s}, and
\cmd{ft-lbm-s}. The output units for these unit systems are shown in
Table 8 (metric) and Table 9 (english). A list of the defined units
abbreviations is given in Table 10.

In addition to the definition of the conversion factors, several
string variables are also defined which describe the output format of
the current units system.  For example, the string variable \cmd{dout}
defines the output format for density units. For the \cmd{in-lbf-sec}
units system, \cmd{dout} =
\cmd{\texttt{"}lbf-sec\^{}2/in\^{}4}\texttt{"}
which is the output format for densities in this system. The string
variables can be used to document the \aprepro{} output. The
string variable names are listed in the last column of Table 8 and Table 9.

\begin{longtable}{l|lllll}
\caption{Units Systems and Corresponding Output Format--Metric} \\
\hline
Quantity      &si&  cgs       & cgs-ev     & shock      & string \\
\hline
Length        & metre    & centimetre & centimetre & centimetre & lout \\
Mass          & kilogram & gram       & gram       & gram       & mout \\
Time          & second   & second     & second     & micro-sec  & tout \\
Temperature   & kelvin   & kelvin     & eV         & kelvin     & Tout \\
Velocity      & metre/sec& cm/sec     & cm/sec     & cm/usec    & vout \\
Acceleration  & metre/sec$^2$ & cm/sec$^2$ & cm/sec$^2$ & cm/usec$^2$ & aout \\
Force         & newton   & dyne       & dyne       & g-cm/usec$^2$ & fout \\
Volume        & metre$^3$ &  cm$^3$ & cm$^3$ & cm$^3$ & Vout \\
Density       & kg/m$^3$ &  g/cc & g/cc & g/cc & dout \\
Energy        & joule    & erg & erg & g-cm$^2$/usec$^3$ & eout \\
Power         & watt     & erg/sec & erg/sec & g-cm$^2$/usec$^4$ & Pout \\
Pressure      & pascal   & dyne/cm$^2$ & dyne/cm$^2$ & Mbar & pout \\
\hline
\end{longtable}


\begin{longtable}{l|llll}
\caption{Units Systems and Corresponding Output Format--English} \\
\hline
Quantity & in-lbf-s & ft-lbf-s & ft-lbm-s & string \\
\hline
Length       & inch         & foot & foot & lout \\
Mass         & lbf-sec$^2$/in & slug & pound-mass & mout \\
Time         & second & second & second & tout \\
Temperature  & rankine & rankine & rankine & Tout \\
Velocity     & inch/sec & foot/sec & foot/sec & vout \\
Acceleration & inch/sec$^2$ & foot/sec$^2$ & foot/sec$^2$ & aout \\
Force        & pound-force & pound-force & poundal & fout \\
Volume       & inch$^3$ & foot$^3$ & foot$^3$ & Vout \\
Density      & lbf-sec$^2$/in$^4$ & slug/ft$^3$ & lbm/ft$^3$ & dout \\
Energy       & inch-lbf & foot-lbf & ft-poundal & eout \\
Power        & inch-lbf/sec & foot-lbf/sec & ft-poundal/sec & Pout\\
Pressure     & lbf/in$^2$ & lbf/ft$^2$ & poundal/ft$^2$ & pout \\
\hline
\end{longtable}

The units definitions are accessed through the \afunc{Units} function in \aprepro{}:
\begin{apinp}
\{Units("\var{unit\_system}")\}
\end{apinp}
where \var{unit\_system} is one of the strings listed in the first row
of the previous two tables. 

\section{Defined Units Variables}

In the following table, the first column lists the variables
that are defined in the \aprepro{} unit system and the second column
is a short description of the unit. All units variables are defined in
terms of the five SI Base Units metre (length), second (time),
kilogram (mass), temperature (kelvin), and radian (angle)\footnote{The
radian is actually a SI Supplementary Unit since it has not been
decided whether it is a Base Unit or a Derived Unit. There are three
other SI Base Units, the candela, ampere, and mole, but they are not
yet used in the Aprepro units system.}. The bolded rows
delineate the type of unit variable and the base quantities used to
define it where $L$ is length, $T$ is time, $M$ is mass, and $t$ is
temperature. For example density is defined in terms of $M/L^3$ which
is mass/ length$^3$.

\begin{longtable}{l|l}
\caption{Defined Units Variables} \\
\hline\hline
\multicolumn{2}{c}{\bf Length \boldmath{[L]}} \\
\hline
m, meter, metre & \cmd{Metre (base unit)} \\
cm, centimeter, centimetre & {Metre / 100} \\
mm, millimeter, millimetre & {Metre / 1,000} \\
um, micrometer, micrometre & {Metre / 1,000,000} \\
km, kilometer, kilometre & {Metre * 1,000} \\
in, inch & Inch   \\
ft, foot & Foot  \\
yd, yard & Yard   \\
mi, mile & Mile   \\
mil & Mil (inch/1000) \\
\multicolumn{2}{c}{} \\
\hline \multicolumn{2}{c}{\bf Time \boldmath{[T]}} \\
\hline
second, sec & \cmd{Second (base unit)} \\
usec, microsecond & Second / 1,000,000 \\
msec, millisecond & Second / 1,000 \\
minute & Minute \\
hr, hour & Hour   \\
day & Day   \\
yr, year & Year = 365.25 days \\
decade & 10 Years \\
century & 100 Years \\
\multicolumn{2}{c}{} \\
\hline 
\multicolumn{2}{c}{\bf Velocity \boldmath{[L/T]}} \\
\hline
mph & Miles per hour \\
kph & Kilometres per hour \\
mps & Metre per second \\
kps & Kilometre per second \\
fps & Foot per second \\
ips & Inch per second \\
\multicolumn{2}{c}{} \\
\hline 
\multicolumn{2}{c}{\bf Acceleration \boldmath{[$L/T^2$]}} \\
\hline
ga & Gravitational acceleration \\
\multicolumn{2}{c}{} \\
\hline 
\multicolumn{2}{c}{\bf Mass \boldmath{[$M$]}} \\
\hline
kg & \cmd{Kilogram (base unit)} \\
g, gram & Gram \\
lbm & Pound (mass) \\
slug & Slug \\
lbfs2pin & Lbf-sec$^2$/in \\
\multicolumn{2}{c}{} \\
\hline 
\multicolumn{2}{c}{\bf Density \boldmath{[$M/L^3]$}} \\
\hline
gpcc & Gram / cm$^3$ \\
kgpm3 & Kilogram / m$^3$ \\
lbfs2pin4 & Lbf-sec$^2$ / in$^4$ \\
lbmpin3 & Lbm / in \\
lbmpft3 & Lbm / ft$^3$ \\
slugpft3 & Slug / ft$^3$ \\
\multicolumn{2}{c}{} \\
\hline 
\multicolumn{2}{c}{\bf Force \boldmath{[$ML/T^2$]}} \\
\hline
N, newton & Newton = 1 kg-m/sec$^2$ \\
dyne & Dyne = newton/10,000 \\
gf & Gram (force) \\
kgf & Kilogram (force) \\
lbf & Pound (force) \\
kip & Kilopound (force) \\
pdl, poundal & Poundal \\
ounce & Ounce = lbf / 16 \\
\multicolumn{2}{c}{} \\
\hline 
\multicolumn{2}{c}{\bf Energy \cmd{[$ML^2/T^2$]}} \\
\hline
J,   joule & Joule = 1 newton-metre \\
ftlbf & Foot-lbf \\
erg & Erg = 1e-7 joule \\
calorie & International Table Calorie \\
Btu & International Table Btu \\
therm & EEC therm \\
tonTNT & Energy in 1 ton TNT \\
kwh & Kilowatt hour \\
\multicolumn{2}{c}{} \\
\hline 
\multicolumn{2}{c}{\bf Power \boldmath{[$ML^2/T^3$]}} \\
\hline
W, watt & Watt = 1 joule / second \\
Hp & Elec. Horsepower (746 W) \\
\multicolumn{2}{c}{} \\
\hline 
\multicolumn{2}{c}{\bf Temperature \boldmath{[t]}} \\
\hline
degK, kelvin & \cmd{Kelvin (Base Unit)} \\
degC & Degree Celsius \\
degF & Degree Fahrenheit \\
degR, rankine & Degree Rankine \\
eV & Electron Volt \\
\multicolumn{2}{c}{} \\
\hline 
\multicolumn{2}{c}{\bf Pressure \boldmath{[$M/L/T^2$]}} \\
\hline
Pa, pascal & Pascal = 1 newton / metre$^2$ \\
MPa & Megapascal \\
GPa & Gigapascal \\
bar & Bar \\
kbar & Kilobar \\
Mbar & Megabar \\
atm & Standard atmosphere \\
torr & Torr = 1 mmHg \\
mHg & Metre of mercury \\
mmHg & Millimetre of mercury \\
inHg & Inch of mercury \\
inH2O & Inch of water \\
ftH2O & Foot of water \\
psi & Pound per square inch \\
ksi & Kilo-pound per square inch \\
psf & Pound per square foot \\
\multicolumn{2}{c}{} \\
\hline 
\multicolumn{2}{c}{\bf Volume \boldmath{[L$^3$]}} \\
\hline
liter & Metre$^3$ / 1000 \\
gal, gallon & Gallon (U.S.) \\
\multicolumn{2}{c}{} \\
\hline 
\multicolumn{2}{c}{\bf Angular }  \\
\hline
rad & \cmd{Radian (base unit)} \\
rev & Full circle = 360 degree \\
deg, degree & Degree \\
arcmin & Arc minute = 1/60 degree \\
arcsec & Arc second = 1/360 degree \\
grade & Grade = 0.9 degree \\
\end{longtable}
The conversion expressions were obtained from
References~\cite{bib:isotopes}, \cite{bib:jaeger}, \cite{bib:lambe}, and~\cite{bib:simpson}.

\section{Usage}

The following example illustrates the basic usage of the \aprepro{} units conversion utility.

\begin{apinp}
\$ Aprepro Units Utility Example 
\$ \{ECHO(OFF)\} \textit{\ldots{}Turn off echoing of the conversion factors}
\$ \{Units(``shock'')\} \textit{\ldots{}Select the shock units system}
\$  NOTE:  Dimensions  -  \{lout\},  \{mout\},  \{dout\},  \{pout\} 
\textit{\ldots{}This will document what quantities are used in the file after it is run through Aprepro}
\{len1  = 10.0 * inch\}  \textit{\ldots{}Define a length in an english unit (inches)}
$ \{len2  = 12.0\textasciitilde{}inch\} \textit{\ldots{}\textasciitilde{} is a synonym for * (multiplication)}
Material  1,  Elastic Plastic,  \{1890\textasciitilde{}kgpm3\}  \$  \{dout\}
   Youngs Modulus = \{28.3e6\textasciitilde{}psi\} \$ {pout}
   Yield  Stress  = \{30\textasciitilde{}ksi\}
   Initial Veclocity = \{10\textasciitilde{}mph\} \$ {vout}
   \textit{\ldots{}Define the density and material parameters in whatever units they are available}
End
Point 100 \{0.0\}  \{0.0\}
Point 110 \{len1\} \{0.0\}
Point 120 \{len1\} \{len2\}
Point 130 \{0.0\}  \{len1\}
\end{apinp}

The output from this example input file is:

\begin{apout}
\$  Aprepro  Units  Utility  Example
\$  NOTE:  Dimensions  -  cm,  gram,  g/cc,  Mbar  \textit{\ldots{}The documentation of what quantities this file uses}
\$ 25.4 
\$ 30.48

Material 1, Elastic Plastic, 1.89 \$ g/cc 
   Youngs Modulus = 1.951216314 \$ Mbar
   Yield  Stress  =  0.002068427188  \textit{\ldots{}All material parameters are now in consistent units}
   Initial Velocity = 0.00044704 \$ cm/usec
End

Point  100  0  0
Point  110  25.4 0
Point  120  25.4 30.48
Point  130  0   25.4   \textit{\ldots{}Lengths have all been converted to centimetres}
\end{apout}

The same input file can be used to output in SI units simply by changing Units 
command from \var{shock} to \var{si}. The output in SI units is:

\begin{apout}
\$ Aprepro Units Utility Example
\$  NOTE:  Dimensions  -  meter,  kilogram,  kg/m^3,  Pa  
\textit{\ldots{}Quantities are now output in standard SI units}
\$ 0.254
\$ 0.3048

Material  1,  Elastic  Plastic,  1890  \$  kg/m^3
   Youngs  Modulus =  1.951216314e+11 \$ Pa
   Yield  Stress   =  206842718.8 
   Initial Velocity = 4.4704 \$ meter/sec
End

Point  100 0  0
Point  110 0.254  0
Point  120 0.254  0.3048
Point  130 0   0.254  \textit{\ldots{}Lengths have all been converted to metres}
\end{apout}

\section{Additional Comments}

A few additional comments and warnings on the use of the units system are detailed 
below.

Omitting the \{ECHO(OFF)\} line prior to the \{Units(``unit\_system'')\} 
function will print out the contents of the units header and conversion files. 
Each line in the output will be preceded by the current comment character which 
is \$ by default.

The comment character can be changed by invoking \aprepro{} with the 
\cmd{-c} option. For example \cmd{aprepro -c\# input\_file output\_file} 
will change the comment character at the beginning of the lines to \#. 

The temperature conversions are only valid for relative temperatures, for 
example, 100\textasciitilde{}degC is equal to 180\textasciitilde{}degF, not 212\textasciitilde{}degF. 

Since several variables are defined in the units system, it is
possible to redefine one of the variable names in your input file. If
the \aprepro{} warning messages are turned off, you will not be
notified of the variable redefinition and erroneous results may
occur. Therefore, you should not turn off \aprepro{} warning
messages while using the units system, and you should investigate all
redefined variable messages to ensure that you are getting the results
you expect. 

