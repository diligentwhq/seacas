\chapter{Functions}\label{ch:functions}

Several mathematical and string functions are implemented in
\aprepro{}.  To cause a function to be used, you enter the name
of the function followed by a list of zero or more arguments in
parentheses. For example
\begin{apinp}
\{sqrt(min(a,b*3))\}
\end{apinp}
uses the two functions \cmd{sqrt()} and \cmd{min()}. The arguments \cmd{a} 
and \cmd{b*3} are passed to \cmd{min()}. The result is then passed as an 
argument to \cmd{sqrt()}. The functions in \aprepro{} are listed below 
along with the number of arguments and a short description of their effect.

\section{Mathematical Functions}

The following mathematical functions are available in \aprepro{}.

\begin{longtable}{lp{4.0in}}
\caption{Mathematical Functions}\label{t:functions}\\
Syntax              & Description \\
\hline
\endhead
abs(x)              &  Absolute value of $x$. $|x|$.\\
acos(x)             &  Inverse cosine of $x$, returns radians.\\
acosd(x)            &  Inverse cosine of $x$, returns degrees.\\
acosh(x)            &  Inverse hyperbolic cosine of $x$.\\
asin(x)             &  Inverse sine of $x$, returns degrees.\\
asin(x)             &  Inverse sine of $x$, returns radians.\\
asinh(x)            &  Inverse hyperbolic sine of $x$.\\
atan(x)             &  Inverse tangent of $x$, returns radians.\\
atan2(x,y)          &  Inverse tangent of $y/x$, returns radians.\\
atan2d(x,y)         &  Inverse tangent of $y/x$, returns degrees.\\
atand(x)            &  Inverse tangent of $x$, returns degrees.\\
atanh(x)            &  Inverse hyperbolic tangent of $x$.\\
ceil(x)             &  Smallest integer not less than $x$.\\
cos(x)              &  Cosine of $x$, with $x$ in radians\\
cosd(x)             &  Cosine of $x$, with $x$ in degrees\\
cosh(x)             &  Hyperbolic cosine of $x$.\\
d2r(x)              &  Degrees to radians.\\
dim(x,y)            &  $x - \min(x,y)$\\
dist(x1,y1, x2,y2)  &  $\sqrt{(x_1-x_2)^2 + (y_1-y_2)^2}$ \\
exp(x)              &  Exponential  $e^x$ \\
floor(x)            &  Largest integer not greater than $x$.\\
fmod(x,y)           &  Floating-point remainder of $x/y$.\\
hypot(x,y)          &  $\sqrt{x^2+y^2}$.\\
int(x), [x]         &  Integer part of $x$ truncated toward 0.\\
julday(mm, dd, yy)  &  Julian day corresponding to mm/dd/yy. \\
juldayhms(mm, dd, yy, hh, mm, ss)&  Julian day corresponding to mm/dd/yy at hh:mm:ss \\
lgamma(x)           &  $\log(\Gamma(x))$.\\
ln(x)               &  Natural (base e) logarithm of $x$.\\
log(x)              &  Natural (base e) logarithm of $x$.\\
log10(x)            &  Base 10 logarithm of $x$. \\
log1p(x)            &  $log(1+x)$ Accurate even for very small values of $x$\\
max(x,y)            &  Maximum of $x$ and $y$. \\
min(x,y)            &  Minimum of $x$ and $y$. \\
nint(x)             &  Rounds $x$ to nearest integer. $<0.5$ down; $>=0.5$ up.\\
polarX(r,a)         &  $r * \cos(a)$, $a$ is in degrees \\
polarY(r,a)         &  $r * \sin(a)$, $a$ is in degrees \\
r2d(x)              &  Radians to degrees. \\
rand(xl,xh)         &  Random value between $xl$ and $xh$; uniformly distributed. \\
rand\_lognormal(m,s)&  Random value with lognormal distribution with mean $m$ and stddev $s$.\\
rand\_normal(m,s)   &  Random value normally distributed with mean $m$ and stddev $s$.\\
rand\_weibull(a, b) &  Random value with weibull distribution with $\alpha=a$ and $\beta=b$. \\
sign(x,y)           &  $x * \text{sgn}(y)$\\
sin(x)              &  Sine of $x$, with $x$ in radians. \\
sind(x)             &  Sine of $x$, with $x$ in degrees. \\
sinh(x)             &  Hyperbolic sine of $x$ \\
sqrt(x)             &  Square root of $x$. $\sqrt{x}$\\
srand(seed)         &  Seed the random number generator with the given integer value. At the beginning of \aprepro{} execution, \cmd{srand()}
                       is called with the current time as the seed. \\
strtod(svar)        &  Returns a double-precision floating-point number equal to the value represented by the character string pointed to by \var{svar}.\\
tan(x)              &  Tangent of $x$, with $x$ in radians. \\
tand(x)             &  Tangent of $x$, with $x$ in radians. \\
tanh(x)             &  Hyperbolic tangent of $x$. \\
Vangle(x1,y1,x2,y2) &  Angle (radians) between vector $x_1\hat{i}+y_1\hat{j}$ and $x_2\hat{i}+y_2\hat{j}$.\\
Vangled(x1,y1,x2,y2)&  Angle (degrees) between vector $x_1\hat{i}+y_1\hat{j}$ and $x_2\hat{i}+y_2\hat{j}$.\\
word\_count(svar,del)&  Number of words in \var{svar}. Words are separated by one or more of the characters in the string variable del.\\
\hline
\end{longtable}

\begin{longtable}{lp{4.0in}}
\caption{String Functions}\label{t:stringfunctions}\\
Syntax              & Description \\
\hline
\endhead
DUMP()              &  Output a list of all defined variables and their value. \\
DUMP\_FUNC()        &  Output a list of all double and string functions recognized by \aprepro{}. \\
DUMP\_PREVAR()      &  Output a list of all predefined variables and their value. \\
IO(x)               &  Convert x to an integer and then to a string. Can be used to output integer values if your output format (\cmd{\_FORMAT}) is set to something that doesn't output integers correctly.  \\
Units(svar)         &  See Chapter~\ref{ch:units}. \var{svar} is one of the defined units systems: 'si', 'cgs', 'cgs-ev', 'shock', 'swap', 'ft-lbf-s', 'ft-lbm-s', 'in-lbf-s' \\
error(svar)         &  Outputs the string \var{svar} to stderr and then terminates the code with an error exit status. \\
execute(svar)       &  \var{svar} is parsed and executed as if it were a line read from the input file. \\
extract(s, b, e)    &  Return substring \var{[b,e)}. \var{b} is included; \var{e} is not. If \var{b} not found, return empty; If \var{e} not found, return rest of string. If \var{b} empty, start at beginning; if \var{e} empty, return rest of string. \\
file\_to\_string(fn)&  Opens the file specified by \var{fn} and returns the contents as a multi-line string. \\
get\_date()         &  Returns a string representing the current date in the form YYYY/MM/DD. \\
get\_iso\_date()    &  Returns a string representing the current date in the form YYYYMMDD. \\
get\_time()         &  Returns a string representing the current time in the form HH:MM:SS. \\
get\_word(n,svar,del&  Returns a string containing the \var{n}th word of \var{svar}. The words are separated by one or more of the characters in the string variable \var{del}  \\
getenv(svar)        &  Returns a string containing the value of the environment variable \var{svar}. If the environment variable is not defined, an empty string is returned.  \\
help()              &  Tell how to get help on variables, functions, \ldots \\
include\_path(path) &  Specify an optional path to be prepended to a filename when opening a file. Can also be specified via the \cmd{-I} command line option when executing aprepro. \\
output(filename)    &  Creates the file specified by filename and
sends all subsequent output from aprepro to that file. Calling \cmd{output(\"stdout\")} will close the current output file and return output to the terminal (standard output).\\
output\_append(fn)  &  If file with name \var{fn} exists, append output to it; otherwise create the file and send all subsequent output from aprepro to that file. \\
rescan(svar)        &  The difference between \cmd{execute(sv1)} and \cmd{rescan(sv2)} is that \var{sv1} must be a valid expression, but \var{sv2} can contain zero or more expressions.  \\
to\_lower(svar)     &  Translates all uppercase characters in \var{svar} to lowercase. It modifies \var{svar} and returns the resulting string.   \\
tolower(svar)       &  Translates all uppercase characters in \var{svar} to lowercase. It modifies \var{svar} and returns the resulting string.   \\
to\_string(x)       &  Returns a string representation of the numerical variable \var{x}. The variable \var{x} is unchanged.  \\
tostring(x)         &  Returns a string representation of the numerical variable \var{x}. The variable \var{x} is unchanged.  \\
to\_upper(svar)     &  Translates all lowercase character in \var{svar} to uppercase. It modifies \var{svar} and returns the resulting string.  \\
toupper(svar)       &  Translates all lowercase character in \var{svar} to uppercase. It modifies \var{svar} and returns the resulting string.  \\
\hline
\end{longtable}

The following example shows the use of some of the string functions. The input:
\begin{apinp}
\{t1 = "ATAN2"\} 
\{t2 = "(0, -1)"\} 
\{t3 = tolower(t1//t2)\} 
\{execute(t3)\} 
\end{apinp}
produces the output:
\begin{apout}
ATAN2 
(0, -1)
atan2(0, -1)   \textit{The variable t3 is equal to the string atan2(0,-1)} 
3.141592654    \textit{The result is the same as executing \{atan2(0, -1)\}}
\end{apout}

This is admittedly a very contrived example; however, it does
illustrate the workings of several of the functions. In the example,
an expression is constructed by concatenating two strings together and
converting the resulting string to lowercase.  This string is then
executed and simply prints the result of evaluating the expression.

The following example uses the \cmd{rescan} function to illustrate a basic
macro capability in \aprepro{}. The example calculates the coordinates
of eleven points (Point1 \ldots{} Point11) equally spaced about the
circumference of a 180 degree arc of radius 10.

\begin{apinp}
\{ECHO(OFF)\}\
\{num = 0\} 
\{rad = 10\} 
\{nintv = 10\} 
\{nloop = nintv + 1\} 
\{line = 'Define \{"Point"//tostring(++num)\}, \{polarX(rad, (num-1) * 
  180/nintv)\} \{polarY(rad, (num-1)*180/nintv)\}'\}
\{ECHO(ON)\} 
\{loop(nloop)\}
\{rescan(line)\} 
\{endloop\}
\end{apinp}

Output:

\begin{apout}
Define  Point1,  10    0
Define  Point2,  9.510565163    3.090169944
Define  Point3,  8.090169944    5.877852523
Define  Point4,  5.877852523    8.090169944
Define  Point5,  3.090169944    9.510565163
Define  Point6,  6.123233765e-16    10
Define  Point7,  -3.090169944    9.510565163
Define  Point8,  -5.877852523    8.090169944
Define  Point9,  -8.090169944    5.877852523
Define  Point10,  -9.510565163    3.090169944
Define  Point11,  -10    1.224646753e-15
\end{apout}

Note the use of the \cmd{ECHO(OFF\textbar{}ON)} block to 
suppress output during the initialization phase, and the loop construct
to automatically repeat the rescan line. The variable \cmd{num} is converted 
to a string after it is incremented and then concatenated to build the name of 
the point. In the definition of the variable \cmd{line}, single quotes are first 
used since this is a multi-line string; double quotes are then used to embed another 
string within the first string. To modify this example to calculate the coordinates 
of 101 points rather than eleven, the only change necessary would be to set \cmd{\{nintv=100\}}.

\section{Additional Functions }

\subsection{[{\em var}] or [{\em expression}]} Surrounding a variable or expression
by square brackets will return the integer value of that variable or
expression truncated toward zero. For example \cmd{[sqrt(2)]} will return the value
\cmd{1}.

\subsection{File Inclusion} \aprepro{} can read input from multiple 
files using the \cmd{include()} and \cmd{cinclude()} functions. If a line 
of the form: 

\cmd{\{include(\texttt{"}\textit{filename}\texttt{"})\}}\\
\cmd{\{include(string\_variable)\}}

is read, \aprepro{} will open and begin reading from the file
\file{filename}.  A string variable can be used as the
argument instead of a literal string value.  When the end of the file
is reached, it will be closed and \aprepro{} will continue
reading from the previous file. The difference between
\cmd{include()} and \cmd{cinclude()} is that if
\file{filename} does not exist, \cmd{include()} will
terminate \aprepro{} with a fatal error, but \cmd{cinclude()} will just
write a warning message and continue with the current file. The
\cmd{cinclude()} function can be thought of as a {\em conditional
include}, that is, include the file if it exists. Multiple include
files are allowed and an included file can also include additional
files. This option can be used to set variables globally in several
files. For example, if two or more input files share common points or
dimensions, those dimensions can be set in one file that is included
in the other files.

If \cmd{ECHO(OFF)} is in effect during in an included file, \cmd{ECHO(ON)} 
will automatically be executed at the end of the included file.

\subsection{Conditionals} Portions of an input file can be conditionally processed 
through the use of the \cmd{if(expression)}, \cmd{elseif(expression)}, \cmd{else}, and \cmd{endif} 
construct.\footnote{The \cmd{Ifdef(expression)} and \cmd{Ifndef(expression)} construct is deprecated. Please use \cmd{if(expression)} and \cmd{if(!expression)} instead.}
The syntax is: 

\begin{apinp}
\{if(expression)\}
\ldots Lines processed if 'expression' is true or non-zero.
\{elseif(expression2)\}
\ldots Lines processed if 'expression' is false and 'expression2' is true.
\{else\}
\ldots Lines processed if both 'expression' and 'expression2' are false.
\{endif\}
\end{apinp}

The \cmd{elseif()} and \cmd{else} are optional.  Note that if \var{expression} is a simple \var{variable}, then its value will be zero or false if it is undefined; a zero value evaluates to false and a non-zero value is true. The \cmd{if} construct can be nested multiple levels.
A warning message will be printed if improper nesting is detected. The \cmd{if(expression)}, 
\cmd{elseif(expression)}, \cmd{else}, and \cmd{endif} are the 
only text parsed on a line. Text that follows these on the same line is
ignored.  For example:

\begin{apinp}
\{if(a > 10 && b < 10)\} This will be ignored no matter what
\ldots Lines processed if \var{a} > 10 and \var{b} < 10.
\{endif\}
\end{apinp}

\subsection{Switch Statements}
The \cmd{switch} statement is a control construct which allows the value of a variable or expression to change the control flow via a multiway branch.
The construct is begun with a \cmd{switch(expression)} statement followed by one or more \cmd{case(expression)} statements and an optional \cmd{default} statement. The construct is ended with an \cmd{endswitch} statement.  The expression in the \cmd{switch(expression)} statement is evaluated and compared to each \cmd{case(expression)} statement in order.  If the values of the two expressions are equal, then the code following that \cmd{case(expression)} is evaluated up to the next \cmd{case()} or \cmd{default} statement. If the expressions in more than one \cmd{case()} match the initial \cmd{switch()} expression, only the first one will be activated.  If none of the \cmd{case()} expressions match the \cmd{switch()} expression, then the code following the \cmd{default} command will be evaluated. An example of the syntax is:

\begin{apinp}
\{a = 10*PI\}
\{switch(10*PI + sin(0))\}
\ldots This is ignored since it is after the switch, but before any \cmd{case()} statements
\{case(1)\}
\ldots This is not executed since \var{1} is not equal to \var{10*PI+sin(0)}
\{case(a)\}
\ldots This is executed since \var{a} matches the value of \var{10*PI+sin(0)}
\{case(10*PI+sin(0))\}
\ldots This is not executed since a previous case was executed.
\{default\}
\ldots This is not executed since a previous case was executed.
\{endswitch\}
\ldots This is executed since the switch construct
 is finished.
\end{apinp}

Switch constructs cannot be nested, but a \cmd{switch()} can be used inside an \cmd{if()} construct and an \cmd{if()} can be used inside a \cmd{case()} construct. 
The \cmd{switch(expression)}, \cmd{case(expression)}, \cmd{default}, and \cmd{endswitch} are the only text parsed on a line. Text that follows these on the same line is
ignored.

\subsection{Loops} Repeated processing of a group of lines can be controlled 
with the \cmd{loop(control)}, and \cmd{endloop} commands. The syntax is:
\begin{apinp}
\{loop(variable)\}
\ldots Process these lines \var{variable} times
\{endloop\}
\end{apinp}

Loops can be nested. A numerical variable or constant must be specified as the 
loop control specifier. You cannot use an algebraic expression such as

\cmd{\{loop(3+5)\}}.

\subsection{ECHO}\label{echo} The printing of lines to the output file can be controlled 
through the use of the \cmd{ECHO(OFF)} and \cmd{ECHO(ON)} 
commands. The syntax is:

\begin{apinp}
\{ECHO(OFF)\}
\ldots These lines will be processed, but not printed to output
\{ECHO(ON)\}
\ldots These lines will be both processed and printed to output.
\end{apinp}

\cmd{ECHO} will automatically be turned on at the end of an included file. The 
commands \cmd{ECHO} and \cmd{NOECHO} are synonyms for \cmd{ECHO(ON)} and \cmd{ECHO(OFF)}.

\subsection{VERBATIM} The printing of all lines to the output file without processing 
can be controlled through the use of the \cmd{VERBATIM(ON)} and \cmd{VERBATIM(OFF)} 
commands. The syntax is:

\begin{apinp}
\{VERBATIM(ON)\}
\ldots These lines will be printed to output, but not processed
\{VERBATIM(OFF)\}
\ldots These lines will be printed to output and processed
\end{apinp}
NOTE:  there  is  a  major  difference  between  the  \cmd{ECHO/NOECHO} commands, 
the \cmd{Ifdef/Endif }commands, and the \cmd{VERBATIM(ON\textbar{}OFF)} commands:

\begin{itemize}
\item \cmd{ECHO(ON\textbar{}OFF)} Lines processed, but not printed if \cmd{ECHO(OFF)}
\item \cmd{Ifdef/Endif} Lines not processed or printed if in \cmd{Ifndef} block
\item \cmd{VERBATIM(ON\textbar{}OFF)} Lines not processed, but are printed.
\end{itemize}

\subsection{IMMUTABLE}\label{immutable_block} Variables can either be
created as mutable or immutable.  By default, all variables created
during a run of aprepro are mutable unless the \cmd{--immutable} or
\cmd{-X} command line option is used to execute \aprepro{}.  An
\cmd{IMMUTABLE} block can also be used to change \aprepro{} such that
all variables are created as immutable.  The syntax is:
\begin{apinp}
\{IMMUTABLE(ON)\}
\ldots All variables created will be immutable
\{IMMUTABLE(OFF)\}
\ldots The mutable/immutable state changes back to the default which
is typically mutable unless \aprepro{} executed with the
\cmd{--immutable} or \cmd{-X} options.
\end{apinp}
Note that any variables created as immutable are still immutable
following the \cmd{IMMUTABLE(OFF)} command.

\subsection{Output File Specification} The \cmd{output} function can be used 
to change the file to which \aprepro{} is outputting the processed
data.  The syntax is: \cmd{\{output("\file{filename}")\}}, where
\file{filename} is the name of the new output file. A string variable
can be used as the function argument. The previous output file is
closed. An error message is written and the code terminates if the
file cannot be opened. If \cmd{output("\file{stdout}")} is specified,
then the current output file is closed and output is again written to
the standard output which is where output is written by default.
